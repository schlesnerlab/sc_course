% Options for packages loaded elsewhere
\PassOptionsToPackage{unicode}{hyperref}
\PassOptionsToPackage{hyphens}{url}
%
\documentclass[
]{article}
\usepackage{amsmath,amssymb}
\usepackage{iftex}
\ifPDFTeX
  \usepackage[T1]{fontenc}
  \usepackage[utf8]{inputenc}
  \usepackage{textcomp} % provide euro and other symbols
\else % if luatex or xetex
  \usepackage{unicode-math} % this also loads fontspec
  \defaultfontfeatures{Scale=MatchLowercase}
  \defaultfontfeatures[\rmfamily]{Ligatures=TeX,Scale=1}
\fi
\usepackage{lmodern}
\ifPDFTeX\else
  % xetex/luatex font selection
\fi
% Use upquote if available, for straight quotes in verbatim environments
\IfFileExists{upquote.sty}{\usepackage{upquote}}{}
\IfFileExists{microtype.sty}{% use microtype if available
  \usepackage[]{microtype}
  \UseMicrotypeSet[protrusion]{basicmath} % disable protrusion for tt fonts
}{}
\makeatletter
\@ifundefined{KOMAClassName}{% if non-KOMA class
  \IfFileExists{parskip.sty}{%
    \usepackage{parskip}
  }{% else
    \setlength{\parindent}{0pt}
    \setlength{\parskip}{6pt plus 2pt minus 1pt}}
}{% if KOMA class
  \KOMAoptions{parskip=half}}
\makeatother
\usepackage{xcolor}
\usepackage[margin=1in]{geometry}
\usepackage{color}
\usepackage{fancyvrb}
\newcommand{\VerbBar}{|}
\newcommand{\VERB}{\Verb[commandchars=\\\{\}]}
\DefineVerbatimEnvironment{Highlighting}{Verbatim}{commandchars=\\\{\}}
% Add ',fontsize=\small' for more characters per line
\usepackage{framed}
\definecolor{shadecolor}{RGB}{248,248,248}
\newenvironment{Shaded}{\begin{snugshade}}{\end{snugshade}}
\newcommand{\AlertTok}[1]{\textcolor[rgb]{0.94,0.16,0.16}{#1}}
\newcommand{\AnnotationTok}[1]{\textcolor[rgb]{0.56,0.35,0.01}{\textbf{\textit{#1}}}}
\newcommand{\AttributeTok}[1]{\textcolor[rgb]{0.13,0.29,0.53}{#1}}
\newcommand{\BaseNTok}[1]{\textcolor[rgb]{0.00,0.00,0.81}{#1}}
\newcommand{\BuiltInTok}[1]{#1}
\newcommand{\CharTok}[1]{\textcolor[rgb]{0.31,0.60,0.02}{#1}}
\newcommand{\CommentTok}[1]{\textcolor[rgb]{0.56,0.35,0.01}{\textit{#1}}}
\newcommand{\CommentVarTok}[1]{\textcolor[rgb]{0.56,0.35,0.01}{\textbf{\textit{#1}}}}
\newcommand{\ConstantTok}[1]{\textcolor[rgb]{0.56,0.35,0.01}{#1}}
\newcommand{\ControlFlowTok}[1]{\textcolor[rgb]{0.13,0.29,0.53}{\textbf{#1}}}
\newcommand{\DataTypeTok}[1]{\textcolor[rgb]{0.13,0.29,0.53}{#1}}
\newcommand{\DecValTok}[1]{\textcolor[rgb]{0.00,0.00,0.81}{#1}}
\newcommand{\DocumentationTok}[1]{\textcolor[rgb]{0.56,0.35,0.01}{\textbf{\textit{#1}}}}
\newcommand{\ErrorTok}[1]{\textcolor[rgb]{0.64,0.00,0.00}{\textbf{#1}}}
\newcommand{\ExtensionTok}[1]{#1}
\newcommand{\FloatTok}[1]{\textcolor[rgb]{0.00,0.00,0.81}{#1}}
\newcommand{\FunctionTok}[1]{\textcolor[rgb]{0.13,0.29,0.53}{\textbf{#1}}}
\newcommand{\ImportTok}[1]{#1}
\newcommand{\InformationTok}[1]{\textcolor[rgb]{0.56,0.35,0.01}{\textbf{\textit{#1}}}}
\newcommand{\KeywordTok}[1]{\textcolor[rgb]{0.13,0.29,0.53}{\textbf{#1}}}
\newcommand{\NormalTok}[1]{#1}
\newcommand{\OperatorTok}[1]{\textcolor[rgb]{0.81,0.36,0.00}{\textbf{#1}}}
\newcommand{\OtherTok}[1]{\textcolor[rgb]{0.56,0.35,0.01}{#1}}
\newcommand{\PreprocessorTok}[1]{\textcolor[rgb]{0.56,0.35,0.01}{\textit{#1}}}
\newcommand{\RegionMarkerTok}[1]{#1}
\newcommand{\SpecialCharTok}[1]{\textcolor[rgb]{0.81,0.36,0.00}{\textbf{#1}}}
\newcommand{\SpecialStringTok}[1]{\textcolor[rgb]{0.31,0.60,0.02}{#1}}
\newcommand{\StringTok}[1]{\textcolor[rgb]{0.31,0.60,0.02}{#1}}
\newcommand{\VariableTok}[1]{\textcolor[rgb]{0.00,0.00,0.00}{#1}}
\newcommand{\VerbatimStringTok}[1]{\textcolor[rgb]{0.31,0.60,0.02}{#1}}
\newcommand{\WarningTok}[1]{\textcolor[rgb]{0.56,0.35,0.01}{\textbf{\textit{#1}}}}
\usepackage{graphicx}
\makeatletter
\def\maxwidth{\ifdim\Gin@nat@width>\linewidth\linewidth\else\Gin@nat@width\fi}
\def\maxheight{\ifdim\Gin@nat@height>\textheight\textheight\else\Gin@nat@height\fi}
\makeatother
% Scale images if necessary, so that they will not overflow the page
% margins by default, and it is still possible to overwrite the defaults
% using explicit options in \includegraphics[width, height, ...]{}
\setkeys{Gin}{width=\maxwidth,height=\maxheight,keepaspectratio}
% Set default figure placement to htbp
\makeatletter
\def\fps@figure{htbp}
\makeatother
\setlength{\emergencystretch}{3em} % prevent overfull lines
\providecommand{\tightlist}{%
  \setlength{\itemsep}{0pt}\setlength{\parskip}{0pt}}
\setcounter{secnumdepth}{-\maxdimen} % remove section numbering
\ifLuaTeX
  \usepackage{selnolig}  % disable illegal ligatures
\fi
\IfFileExists{bookmark.sty}{\usepackage{bookmark}}{\usepackage{hyperref}}
\IfFileExists{xurl.sty}{\usepackage{xurl}}{} % add URL line breaks if available
\urlstyle{same}
\hypersetup{
  pdftitle={Functional Analyses},
  hidelinks,
  pdfcreator={LaTeX via pandoc}}

\title{Functional Analyses}
\author{}
\date{\vspace{-2.5em}}

\begin{document}
\maketitle

{
\setcounter{tocdepth}{2}
\tableofcontents
}
\hypertarget{setup}{%
\section{Setup}\label{setup}}

\begin{Shaded}
\begin{Highlighting}[]
\CommentTok{\# .libPaths(new = "/scratch/local/rseurat/pkg{-}lib{-}4.1.3")}

\FunctionTok{suppressMessages}\NormalTok{(\{}
  \FunctionTok{library}\NormalTok{(decoupleR)}
  \FunctionTok{library}\NormalTok{(tidyverse)}
  \FunctionTok{library}\NormalTok{(Seurat)}
  \FunctionTok{library}\NormalTok{(pheatmap)}
  \FunctionTok{library}\NormalTok{(SCpubr)}
\NormalTok{\})}


\FunctionTok{set.seed}\NormalTok{(}\DecValTok{8211673}\NormalTok{)}

\NormalTok{knitr}\SpecialCharTok{::}\NormalTok{opts\_chunk}\SpecialCharTok{$}\FunctionTok{set}\NormalTok{(}\AttributeTok{echo =} \ConstantTok{TRUE}\NormalTok{, }\AttributeTok{format =} \ConstantTok{TRUE}\NormalTok{, }\AttributeTok{out.width =} \StringTok{"100\%"}\NormalTok{)}


\FunctionTok{options}\NormalTok{(}
  \AttributeTok{parallelly.fork.enable =} \ConstantTok{FALSE}\NormalTok{,}
  \AttributeTok{future.globals.maxSize =} \DecValTok{8} \SpecialCharTok{*} \DecValTok{1024}\SpecialCharTok{\^{}}\DecValTok{2} \SpecialCharTok{*} \DecValTok{1000}
\NormalTok{)}

\FunctionTok{plan}\NormalTok{(}\StringTok{"multicore"}\NormalTok{, }\AttributeTok{workers =} \DecValTok{8}\NormalTok{)}
\end{Highlighting}
\end{Shaded}

\begin{verbatim}
## work directory:  /omics/groups/OE0436/internal/heyer/scourse/sc_course/Rmd Files/Day2
\end{verbatim}

\begin{verbatim}
## library path(s):  /omics/groups/OE0436/internal/heyer/scourse/sc_course/renv/library/R-4.3/x86_64-pc-linux-gnu /home/heyer/.cache/R/renv/sandbox/R-4.3/x86_64-pc-linux-gnu/587c2cfa
\end{verbatim}

\hypertarget{load-data}{%
\section{Load Data}\label{load-data}}

We'll be working with the data from our past notebook (``First steps''),
let's quickly re-load and re-process again:

\begin{Shaded}
\begin{Highlighting}[]
\NormalTok{pbmc }\OtherTok{\textless{}{-}} \FunctionTok{Read10X}\NormalTok{(}\AttributeTok{data.dir =} \StringTok{"./datasets/filtered\_gene\_bc\_matrices/hg19/"}\NormalTok{) }\SpecialCharTok{\%\textgreater{}\%}
  \FunctionTok{CreateSeuratObject}\NormalTok{(}\AttributeTok{counts =}\NormalTok{ ., }\AttributeTok{project =} \StringTok{"pbmc3k"}\NormalTok{, }\AttributeTok{min.cells =} \DecValTok{3}\NormalTok{, }\AttributeTok{min.features =} \DecValTok{200}\NormalTok{)}

\NormalTok{pbmc[[}\StringTok{"percent.mt"}\NormalTok{]] }\OtherTok{\textless{}{-}} \FunctionTok{PercentageFeatureSet}\NormalTok{(pbmc, }\AttributeTok{pattern =} \StringTok{"\^{}MT{-}"}\NormalTok{)}

\NormalTok{pbmc }\OtherTok{\textless{}{-}} \FunctionTok{subset}\NormalTok{(pbmc, }\AttributeTok{subset =}\NormalTok{ nFeature\_RNA }\SpecialCharTok{\textgreater{}} \DecValTok{200} \SpecialCharTok{\&}\NormalTok{ nFeature\_RNA }\SpecialCharTok{\textless{}} \DecValTok{2500} \SpecialCharTok{\&}\NormalTok{ percent.mt }\SpecialCharTok{\textless{}} \DecValTok{5}\NormalTok{)}

\NormalTok{pbmc }\OtherTok{\textless{}{-}} \FunctionTok{NormalizeData}\NormalTok{(pbmc, }\AttributeTok{verbose =} \ConstantTok{FALSE}\NormalTok{)}

\NormalTok{pbmc }\OtherTok{\textless{}{-}} \FunctionTok{FindVariableFeatures}\NormalTok{(pbmc, }\AttributeTok{verbose =} \ConstantTok{FALSE}\NormalTok{)}

\NormalTok{pbmc }\OtherTok{\textless{}{-}} \FunctionTok{ScaleData}\NormalTok{(pbmc, }\AttributeTok{features =} \FunctionTok{rownames}\NormalTok{(pbmc), }\AttributeTok{verbose =} \ConstantTok{FALSE}\NormalTok{)}

\NormalTok{pbmc }\OtherTok{\textless{}{-}} \FunctionTok{RunPCA}\NormalTok{(pbmc, }\AttributeTok{features =} \FunctionTok{VariableFeatures}\NormalTok{(pbmc), }\AttributeTok{verbose =} \ConstantTok{FALSE}\NormalTok{)}

\NormalTok{pbmc }\OtherTok{\textless{}{-}} \FunctionTok{FindNeighbors}\NormalTok{(pbmc, }\AttributeTok{dims =} \FunctionTok{seq\_len}\NormalTok{(}\DecValTok{10}\NormalTok{), }\AttributeTok{verbose =} \ConstantTok{FALSE}\NormalTok{)}

\NormalTok{pbmc }\OtherTok{\textless{}{-}} \FunctionTok{FindClusters}\NormalTok{(pbmc, }\AttributeTok{resolution =} \FloatTok{0.5}\NormalTok{, }\AttributeTok{verbose =} \ConstantTok{FALSE}\NormalTok{)}

\NormalTok{pbmc }\OtherTok{\textless{}{-}} \FunctionTok{RunUMAP}\NormalTok{(pbmc, }\AttributeTok{dims =} \FunctionTok{seq\_len}\NormalTok{(}\DecValTok{10}\NormalTok{), }\AttributeTok{verbose =} \ConstantTok{FALSE}\NormalTok{)}
\end{Highlighting}
\end{Shaded}

Next, we do our between-clusters DE analysis:

\begin{Shaded}
\begin{Highlighting}[]
\NormalTok{sample }\OtherTok{\textless{}{-}} \FunctionTok{readRDS}\NormalTok{(}\StringTok{"../sample\_processed.RDS"}\NormalTok{)}
\NormalTok{markers.between.clusters }\OtherTok{\textless{}{-}} \FunctionTok{FindAllMarkers}\NormalTok{(}
\NormalTok{  sample,}
  \AttributeTok{test.use =} \StringTok{"wilcox"}\NormalTok{,}
  \AttributeTok{logfc.threshold =} \FloatTok{0.5}\NormalTok{,}
  \AttributeTok{min.pct =} \FloatTok{0.3}\NormalTok{,}
  \AttributeTok{only.pos =} \ConstantTok{TRUE}\NormalTok{,}
  \AttributeTok{densify =} \ConstantTok{TRUE}
\NormalTok{)}
\end{Highlighting}
\end{Shaded}

\begin{verbatim}
## Calculating cluster 0
\end{verbatim}

\begin{verbatim}
## Calculating cluster 1
\end{verbatim}

\begin{verbatim}
## Calculating cluster 2
\end{verbatim}

\begin{verbatim}
## Calculating cluster 3
\end{verbatim}

\begin{verbatim}
## Calculating cluster 4
\end{verbatim}

\begin{verbatim}
## Calculating cluster 5
\end{verbatim}

\begin{verbatim}
## Calculating cluster 6
\end{verbatim}

\begin{verbatim}
## Calculating cluster 7
\end{verbatim}

\begin{verbatim}
## Calculating cluster 8
\end{verbatim}

\begin{verbatim}
## Calculating cluster 9
\end{verbatim}

\begin{verbatim}
## Calculating cluster 10
\end{verbatim}

\hypertarget{functional-enrichment-analysis}{%
\section{Functional Enrichment
Analysis}\label{functional-enrichment-analysis}}

These methods have first been used for microarrays, and aim to draw
conclusions ranked gene list from RNAseq experiments, scRNA, or any
other OMICS screen. There are a number of tools and approaches - here we
will focus only one common and practical approach.

For more information, see:
\href{https://yulab-smu.top/biomedical-knowledge-mining-book/clusterProfiler-dplyr.html}{clusterProfiler}.

The aim is to draw conclusions as to what's the functional implications
that we may be able to derive given a list of genes. To this end, we'd
start with such list and then consult databases for the annotations.
With this data, we can come up with scores to measure level of
association. A gene set is an unordered collection of genes that are
functionally related.

\hypertarget{gene-ontology}{%
\subsection{Gene Ontology}\label{gene-ontology}}

\href{http://www.geneontology.org/}{GO Terms} are semantic
representations in a curated database that defines concepts/classes used
to describe gene function, and relationships between these concepts. GO
terms are organized in a directed acyclic graph, where edges between
terms represent parent-child relationship. It classifies functions along
three aspects:

\begin{itemize}
\tightlist
\item
  MF: Molecular Function, molecular activities of gene products
\item
  CC: Cellular Compartment, where gene products are active
\item
  BP: Biological Processes, pathways and larger processes made up of the
  activities of multiple gene products
\end{itemize}

\hypertarget{enrichr}{%
\subsection{enrichR}\label{enrichr}}

The package is already loaded. But we need to select which databases to
connect. There are more than 200 databases available, you can get a data
frame with details of these using \texttt{listEnrichrDbs()}. For this
and the following to work you need a working internet connection.

\begin{Shaded}
\begin{Highlighting}[]
\NormalTok{decoupleR}\SpecialCharTok{::}\FunctionTok{get\_resource}\NormalTok{(}\StringTok{"PanglaoDB"}\NormalTok{) }\OtherTok{{-}\textgreater{}}\NormalTok{ panglao\_data}

\NormalTok{panglao\_data }\SpecialCharTok{\%\textgreater{}\%}\NormalTok{ dplyr}\SpecialCharTok{::}\FunctionTok{mutate}\NormalTok{(}\AttributeTok{human =} \FunctionTok{as.logical}\NormalTok{(human)) }\SpecialCharTok{\%\textgreater{}\%}\NormalTok{dplyr}\SpecialCharTok{::}\FunctionTok{filter}\NormalTok{(human) }\SpecialCharTok{\%\textgreater{}\%}\NormalTok{ dplyr}\SpecialCharTok{::}\FunctionTok{mutate}\NormalTok{(}\AttributeTok{canonical\_marker =} \FunctionTok{as.logical}\NormalTok{(canonical\_marker)) }\SpecialCharTok{\%\textgreater{}\%}\NormalTok{ dplyr}\SpecialCharTok{::}\FunctionTok{filter}\NormalTok{(canonical\_marker) }\SpecialCharTok{\%\textgreater{}\%}\NormalTok{ dplyr}\SpecialCharTok{::}\FunctionTok{filter}\NormalTok{(}\StringTok{"human\_sensitivity"} \SpecialCharTok{\textgreater{}} \FloatTok{0.5}\NormalTok{) }\OtherTok{{-}\textgreater{}}\NormalTok{ panglao\_human\_marker}
\end{Highlighting}
\end{Shaded}

\begin{Shaded}
\begin{Highlighting}[]
\NormalTok{mat }\OtherTok{\textless{}{-}} \FunctionTok{as.matrix}\NormalTok{(sample}\SpecialCharTok{@}\NormalTok{assays}\SpecialCharTok{$}\NormalTok{RNA}\SpecialCharTok{$}\NormalTok{data)}
\end{Highlighting}
\end{Shaded}

\begin{verbatim}
## Warning in asMethod(object): sparse->dense coercion: allocating vector of size 2.1 GiB
\end{verbatim}

\begin{Shaded}
\begin{Highlighting}[]
\NormalTok{panglao\_human\_marker}\OtherTok{\textless{}{-}}\NormalTok{ panglao\_human\_marker[}\SpecialCharTok{!}\FunctionTok{duplicated}\NormalTok{(panglao\_human\_marker[,}\FunctionTok{c}\NormalTok{(}\StringTok{"cell\_type"}\NormalTok{, }\StringTok{"genesymbol"}\NormalTok{)]),]}
\NormalTok{ora\_enrich }\OtherTok{\textless{}{-}}\NormalTok{ decoupleR}\SpecialCharTok{::}\FunctionTok{run\_aucell}\NormalTok{(}
    \AttributeTok{mat=}\NormalTok{mat,}
    \AttributeTok{network=}\NormalTok{panglao\_human\_marker,}
    \AttributeTok{.source=}\StringTok{\textquotesingle{}cell\_type\textquotesingle{}}\NormalTok{,}
    \AttributeTok{.target=}\StringTok{\textquotesingle{}genesymbol\textquotesingle{}}\NormalTok{,}
    \AttributeTok{minsize=}\DecValTok{3}\NormalTok{,}
\NormalTok{)}
\end{Highlighting}
\end{Shaded}

\begin{verbatim}
## Warning in .AUCell_buildRankings(exprMat = exprMat, featureType = featureType, : nCores is no longer used. It will be
## deprecated in the next AUCell version.
\end{verbatim}

\begin{Shaded}
\begin{Highlighting}[]
\NormalTok{sample[[}\StringTok{"panglao\_aucell"}\NormalTok{]] }\OtherTok{\textless{}{-}}\NormalTok{   ora\_enrich }\SpecialCharTok{\%\textgreater{}\%}
  \FunctionTok{pivot\_wider}\NormalTok{(}\AttributeTok{id\_cols =} \StringTok{\textquotesingle{}source\textquotesingle{}}\NormalTok{, }\AttributeTok{names\_from =} \StringTok{\textquotesingle{}condition\textquotesingle{}}\NormalTok{,}
              \AttributeTok{values\_from =} \StringTok{\textquotesingle{}score\textquotesingle{}}\NormalTok{) }\SpecialCharTok{\%\textgreater{}\%}
  \FunctionTok{column\_to\_rownames}\NormalTok{(}\StringTok{\textquotesingle{}source\textquotesingle{}}\NormalTok{) }\SpecialCharTok{\%\textgreater{}\%}
\NormalTok{  Seurat}\SpecialCharTok{::}\FunctionTok{CreateAssayObject}\NormalTok{(.)}
\CommentTok{\# Scale the data}
\NormalTok{sample }\OtherTok{\textless{}{-}} \FunctionTok{ScaleData}\NormalTok{(sample, }\AttributeTok{assay =} \StringTok{"panglao\_aucell"}\NormalTok{)}
\end{Highlighting}
\end{Shaded}

\begin{verbatim}
## Centering and scaling data matrix
\end{verbatim}

\begin{Shaded}
\begin{Highlighting}[]
\NormalTok{sample}\SpecialCharTok{@}\NormalTok{assays}\SpecialCharTok{$}\NormalTok{panglao\_ora}\SpecialCharTok{$}\NormalTok{data }\OtherTok{\textless{}{-}}\NormalTok{ sample}\SpecialCharTok{@}\NormalTok{assays}\SpecialCharTok{$}\NormalTok{panglao\_ora}\SpecialCharTok{$}\NormalTok{scale.data}
\end{Highlighting}
\end{Shaded}

\begin{Shaded}
\begin{Highlighting}[]
\NormalTok{n\_tfs }\OtherTok{\textless{}{-}} \DecValTok{3}
\CommentTok{\# Extract activities from object as a long dataframe}
\NormalTok{df }\OtherTok{\textless{}{-}} \FunctionTok{t}\NormalTok{(}\FunctionTok{as.matrix}\NormalTok{(sample}\SpecialCharTok{@}\NormalTok{assays}\SpecialCharTok{$}\NormalTok{panglao\_aucell}\SpecialCharTok{@}\NormalTok{scale.data)) }\SpecialCharTok{\%\textgreater{}\%}
  \FunctionTok{as.data.frame}\NormalTok{() }\SpecialCharTok{\%\textgreater{}\%}
  \FunctionTok{mutate}\NormalTok{(}\AttributeTok{cluster =} \FunctionTok{Idents}\NormalTok{(sample)) }\SpecialCharTok{\%\textgreater{}\%}
  \FunctionTok{pivot\_longer}\NormalTok{(}\AttributeTok{cols =} \SpecialCharTok{{-}}\NormalTok{cluster, }\AttributeTok{names\_to =} \StringTok{"source"}\NormalTok{, }\AttributeTok{values\_to =} \StringTok{"score"}\NormalTok{) }\SpecialCharTok{\%\textgreater{}\%}
  \FunctionTok{group\_by}\NormalTok{(cluster, source) }\SpecialCharTok{\%\textgreater{}\%}
  \FunctionTok{summarise}\NormalTok{(}\AttributeTok{mean =} \FunctionTok{mean}\NormalTok{(score))}
\end{Highlighting}
\end{Shaded}

\begin{verbatim}
## `summarise()` has grouped output by 'cluster'. You can override using the `.groups` argument.
\end{verbatim}

\begin{Shaded}
\begin{Highlighting}[]
\NormalTok{top\_3\_cell\_types }\OtherTok{\textless{}{-}}\NormalTok{ df }\SpecialCharTok{\%\textgreater{}\%}
    \FunctionTok{group\_by}\NormalTok{(cluster) }\SpecialCharTok{\%\textgreater{}\%}
    \FunctionTok{top\_n}\NormalTok{(}\DecValTok{3}\NormalTok{, mean)}

\NormalTok{all\_top\_cell\_types }\OtherTok{\textless{}{-}}\NormalTok{ top\_3\_cell\_types }\SpecialCharTok{\%\textgreater{}\%} \FunctionTok{pull}\NormalTok{(source) }\SpecialCharTok{\%\textgreater{}\%} \FunctionTok{unique}\NormalTok{()}

\NormalTok{top\_acts\_mat }\OtherTok{\textless{}{-}}\NormalTok{ df }\SpecialCharTok{\%\textgreater{}\%}
  \FunctionTok{filter}\NormalTok{(source }\SpecialCharTok{\%in\%}\NormalTok{ all\_top\_cell\_types) }\SpecialCharTok{\%\textgreater{}\%}
  \FunctionTok{pivot\_wider}\NormalTok{(}\AttributeTok{id\_cols =} \StringTok{\textquotesingle{}cluster\textquotesingle{}}\NormalTok{, }\AttributeTok{names\_from =} \StringTok{\textquotesingle{}source\textquotesingle{}}\NormalTok{,}
              \AttributeTok{values\_from =} \StringTok{\textquotesingle{}mean\textquotesingle{}}\NormalTok{) }\SpecialCharTok{\%\textgreater{}\%}
  \FunctionTok{column\_to\_rownames}\NormalTok{(}\StringTok{\textquotesingle{}cluster\textquotesingle{}}\NormalTok{) }\SpecialCharTok{\%\textgreater{}\%}
  \FunctionTok{as.matrix}\NormalTok{()}

\CommentTok{\# Choose color palette}
\NormalTok{palette\_length }\OtherTok{=} \DecValTok{100}
\NormalTok{my\_color }\OtherTok{=} \FunctionTok{colorRampPalette}\NormalTok{(}\FunctionTok{c}\NormalTok{(}\StringTok{"Darkblue"}\NormalTok{, }\StringTok{"white"}\NormalTok{,}\StringTok{"red"}\NormalTok{))(palette\_length)}

\NormalTok{my\_breaks }\OtherTok{\textless{}{-}} \FunctionTok{c}\NormalTok{(}\FunctionTok{seq}\NormalTok{(}\SpecialCharTok{{-}}\DecValTok{3}\NormalTok{, }\DecValTok{0}\NormalTok{, }\AttributeTok{length.out=}\FunctionTok{ceiling}\NormalTok{(palette\_length}\SpecialCharTok{/}\DecValTok{2}\NormalTok{) }\SpecialCharTok{+} \DecValTok{1}\NormalTok{),}
               \FunctionTok{seq}\NormalTok{(}\FloatTok{0.05}\NormalTok{, }\DecValTok{3}\NormalTok{, }\AttributeTok{length.out=}\FunctionTok{floor}\NormalTok{(palette\_length}\SpecialCharTok{/}\DecValTok{2}\NormalTok{)))}

\CommentTok{\# Plot}
\FunctionTok{pheatmap}\NormalTok{(top\_acts\_mat, }\AttributeTok{border\_color =} \ConstantTok{NA}\NormalTok{,}\AttributeTok{scale =} \StringTok{"row"}\NormalTok{, }\AttributeTok{color=}\NormalTok{my\_color, }\AttributeTok{breaks =}\NormalTok{ my\_breaks)}
\end{Highlighting}
\end{Shaded}

\includegraphics[width=1\linewidth]{23_Enrich_files/figure-latex/unnamed-chunk-29-1}

\begin{quote}
⌨🔥 Exercise(s):

\begin{enumerate}
\def\labelenumi{\arabic{enumi}.}
\tightlist
\item
  Understand the format and interpret the output \texttt{result.1}.
\item
  What is the most significantly enriched molecular function? which
  genes are the base for it?
\item
  Would you get the same result if you changed the number of marker
  genes in the input? Try it out.
\end{enumerate}
\end{quote}

We can also produce nice graphical summaries:

\begin{Shaded}
\begin{Highlighting}[]
\NormalTok{net }\OtherTok{\textless{}{-}} \FunctionTok{get\_progeny}\NormalTok{(}\AttributeTok{organism =} \StringTok{\textquotesingle{}human\textquotesingle{}}\NormalTok{, }\AttributeTok{top =} \DecValTok{500}\NormalTok{)}
\NormalTok{net}
\end{Highlighting}
\end{Shaded}

\begin{verbatim}
## # A tibble: 7,000 x 4
##    source   target  weight  p_value
##    <chr>    <chr>    <dbl>    <dbl>
##  1 Androgen TMPRSS2  11.5  2.38e-47
##  2 Androgen NKX3-1   10.6  2.21e-44
##  3 Androgen MBOAT2   10.5  4.63e-44
##  4 Androgen KLK2     10.2  1.94e-40
##  5 Androgen SARG     11.4  2.79e-40
##  6 Androgen SLC38A4   7.36 1.25e-39
##  7 Androgen MTMR9     6.13 2.53e-38
##  8 Androgen ZBTB16   10.6  1.57e-36
##  9 Androgen KCNN2     9.47 7.71e-36
## 10 Androgen OPRK1    -5.63 1.11e-35
## # i 6,990 more rows
\end{verbatim}

\begin{Shaded}
\begin{Highlighting}[]
\NormalTok{mat }\OtherTok{\textless{}{-}} \FunctionTok{as.matrix}\NormalTok{(sample}\SpecialCharTok{@}\NormalTok{assays}\SpecialCharTok{$}\NormalTok{RNA}\SpecialCharTok{@}\NormalTok{data)}
\end{Highlighting}
\end{Shaded}

\begin{verbatim}
## Warning in asMethod(object): sparse->dense coercion: allocating vector of size 2.1 GiB
\end{verbatim}

\begin{Shaded}
\begin{Highlighting}[]
\CommentTok{\# Run mlm}
\NormalTok{acts }\OtherTok{\textless{}{-}} \FunctionTok{run\_mlm}\NormalTok{(}\AttributeTok{mat=}\NormalTok{mat, }\AttributeTok{network =}\NormalTok{net, }\AttributeTok{.source=}\StringTok{\textquotesingle{}source\textquotesingle{}}\NormalTok{, }\AttributeTok{.target=}\StringTok{\textquotesingle{}target\textquotesingle{}}\NormalTok{,}
                \AttributeTok{.mor=}\StringTok{\textquotesingle{}weight\textquotesingle{}}\NormalTok{, }\AttributeTok{minsize =} \DecValTok{5}\NormalTok{)}
\NormalTok{acts}
\end{Highlighting}
\end{Shaded}

\begin{verbatim}
## # A tibble: 116,004 x 5
##    statistic source   condition                              score p_value
##    <chr>     <chr>    <chr>                                  <dbl>   <dbl>
##  1 mlm       Androgen OE0145-IDH_NCH6341_AAACCCAAGGTCGTGA-1  1.59  0.111  
##  2 mlm       EGFR     OE0145-IDH_NCH6341_AAACCCAAGGTCGTGA-1 -2.27  0.0233 
##  3 mlm       Estrogen OE0145-IDH_NCH6341_AAACCCAAGGTCGTGA-1  1.71  0.0866 
##  4 mlm       Hypoxia  OE0145-IDH_NCH6341_AAACCCAAGGTCGTGA-1  2.59  0.00949
##  5 mlm       JAK-STAT OE0145-IDH_NCH6341_AAACCCAAGGTCGTGA-1 -1.68  0.0934 
##  6 mlm       MAPK     OE0145-IDH_NCH6341_AAACCCAAGGTCGTGA-1 -1.99  0.0467 
##  7 mlm       NFkB     OE0145-IDH_NCH6341_AAACCCAAGGTCGTGA-1 -0.883 0.377  
##  8 mlm       PI3K     OE0145-IDH_NCH6341_AAACCCAAGGTCGTGA-1 -2.55  0.0108 
##  9 mlm       TGFb     OE0145-IDH_NCH6341_AAACCCAAGGTCGTGA-1  1.19  0.234  
## 10 mlm       TNFa     OE0145-IDH_NCH6341_AAACCCAAGGTCGTGA-1  0.355 0.723  
## # i 115,994 more rows
\end{verbatim}

\begin{Shaded}
\begin{Highlighting}[]
\CommentTok{\# Extract mlm and store it in pathwaysmlm in data}
\NormalTok{sample[[}\StringTok{\textquotesingle{}pathwaysmlm\textquotesingle{}}\NormalTok{]] }\OtherTok{\textless{}{-}}\NormalTok{ acts }\SpecialCharTok{\%\textgreater{}\%}
  \FunctionTok{pivot\_wider}\NormalTok{(}\AttributeTok{id\_cols =} \StringTok{\textquotesingle{}source\textquotesingle{}}\NormalTok{, }\AttributeTok{names\_from =} \StringTok{\textquotesingle{}condition\textquotesingle{}}\NormalTok{,}
              \AttributeTok{values\_from =} \StringTok{\textquotesingle{}score\textquotesingle{}}\NormalTok{) }\SpecialCharTok{\%\textgreater{}\%}
  \FunctionTok{column\_to\_rownames}\NormalTok{(}\StringTok{\textquotesingle{}source\textquotesingle{}}\NormalTok{) }\SpecialCharTok{\%\textgreater{}\%}
\NormalTok{  Seurat}\SpecialCharTok{::}\FunctionTok{CreateAssayObject}\NormalTok{(.)}

\CommentTok{\# Change assay}
\FunctionTok{DefaultAssay}\NormalTok{(}\AttributeTok{object =}\NormalTok{ sample) }\OtherTok{\textless{}{-}} \StringTok{"pathwaysmlm"}

\CommentTok{\# Scale the data}
\NormalTok{sample }\OtherTok{\textless{}{-}} \FunctionTok{ScaleData}\NormalTok{(sample, }\AttributeTok{assay =} \StringTok{"pathwaysmlm"}\NormalTok{)}
\end{Highlighting}
\end{Shaded}

\begin{verbatim}
## Centering and scaling data matrix
\end{verbatim}

\begin{Shaded}
\begin{Highlighting}[]
\NormalTok{sample}\SpecialCharTok{@}\NormalTok{assays}\SpecialCharTok{$}\NormalTok{pathwaysmlm}\SpecialCharTok{@}\NormalTok{data }\OtherTok{\textless{}{-}}\NormalTok{ sample}\SpecialCharTok{@}\NormalTok{assays}\SpecialCharTok{$}\NormalTok{pathwaysmlm}\SpecialCharTok{@}\NormalTok{scale.data}
\end{Highlighting}
\end{Shaded}

\begin{Shaded}
\begin{Highlighting}[]
\NormalTok{p1 }\OtherTok{\textless{}{-}} \FunctionTok{DimPlot}\NormalTok{(sample, }\AttributeTok{reduction =} \StringTok{"umap"}\NormalTok{, }\AttributeTok{label =} \ConstantTok{TRUE}\NormalTok{, }\AttributeTok{pt.size =} \FloatTok{0.5}\NormalTok{) }\SpecialCharTok{+} 
  \FunctionTok{NoLegend}\NormalTok{() }\SpecialCharTok{+} \FunctionTok{ggtitle}\NormalTok{(}\StringTok{\textquotesingle{}Cell types\textquotesingle{}}\NormalTok{)}
\NormalTok{p2 }\OtherTok{\textless{}{-}}\NormalTok{ (}\FunctionTok{do\_NebulosaPlot}\NormalTok{(sample, }\AttributeTok{features =} \FunctionTok{c}\NormalTok{(}\StringTok{"MAPK"}\NormalTok{),) }\SpecialCharTok{\&} 
  \FunctionTok{scale\_colour\_gradient2}\NormalTok{(}\AttributeTok{low =} \StringTok{\textquotesingle{}blue\textquotesingle{}}\NormalTok{, }\AttributeTok{mid =} \StringTok{\textquotesingle{}white\textquotesingle{}}\NormalTok{, }\AttributeTok{high =} \StringTok{\textquotesingle{}red\textquotesingle{}}\NormalTok{)) }\SpecialCharTok{+}
  \FunctionTok{ggtitle}\NormalTok{(}\StringTok{\textquotesingle{}MAPK activity\textquotesingle{}}\NormalTok{)}
\end{Highlighting}
\end{Shaded}

\begin{verbatim}
## Warning in ks.defaults(x = x, w = w, binned = binned, bgridsize = bgridsize, : Weights don't sum to sample size - they have been scaled accordingly
\end{verbatim}

\begin{verbatim}
## Scale for colour is already present.
## Adding another scale for colour, which will replace the existing scale.
\end{verbatim}

\begin{Shaded}
\begin{Highlighting}[]
\NormalTok{p1 }\SpecialCharTok{|}\NormalTok{ p2}
\end{Highlighting}
\end{Shaded}

\includegraphics[width=1\linewidth]{23_Enrich_files/figure-latex/unnamed-chunk-32-1}
\#\#\# Compare with results from Publication

\begin{Shaded}
\begin{Highlighting}[]
\NormalTok{excel\_data }\OtherTok{\textless{}{-}}\NormalTok{ readxl}\SpecialCharTok{::}\FunctionTok{read\_excel}\NormalTok{(}\StringTok{"../1{-}s2.0{-}S2666379123004263{-}mmc3.xlsx"}\NormalTok{, }\AttributeTok{sheet =} \StringTok{"OD top50 markers"}\NormalTok{,}\AttributeTok{skip =} \DecValTok{1}\NormalTok{)}

\NormalTok{excel\_data }\SpecialCharTok{\%\textgreater{}\%} \FunctionTok{pivot\_longer}\NormalTok{(}\AttributeTok{cols =} \FunctionTok{everything}\NormalTok{()) }\OtherTok{{-}\textgreater{}}\NormalTok{ excel\_long}

\NormalTok{acts\_paper }\OtherTok{\textless{}{-}}\NormalTok{ decoupleR}\SpecialCharTok{::}\FunctionTok{run\_aucell}\NormalTok{(mat, }\AttributeTok{network =}\NormalTok{ excel\_long, }\AttributeTok{.source =} \StringTok{"name"}\NormalTok{, }\AttributeTok{.target =} \StringTok{"value"}\NormalTok{)}
\end{Highlighting}
\end{Shaded}

\begin{verbatim}
## Warning in .AUCell_buildRankings(exprMat = exprMat, featureType = featureType, : nCores is no longer used. It will be
## deprecated in the next AUCell version.
\end{verbatim}

\begin{Shaded}
\begin{Highlighting}[]
\CommentTok{\# Extract mlm and store it in pathwaysmlm in data}
\NormalTok{sample[[}\StringTok{\textquotesingle{}paper\_aucell\textquotesingle{}}\NormalTok{]] }\OtherTok{\textless{}{-}}\NormalTok{ acts\_paper }\SpecialCharTok{\%\textgreater{}\%}
  \FunctionTok{pivot\_wider}\NormalTok{(}\AttributeTok{id\_cols =} \StringTok{\textquotesingle{}source\textquotesingle{}}\NormalTok{, }\AttributeTok{names\_from =} \StringTok{\textquotesingle{}condition\textquotesingle{}}\NormalTok{,}
              \AttributeTok{values\_from =} \StringTok{\textquotesingle{}score\textquotesingle{}}\NormalTok{) }\SpecialCharTok{\%\textgreater{}\%}
  \FunctionTok{column\_to\_rownames}\NormalTok{(}\StringTok{\textquotesingle{}source\textquotesingle{}}\NormalTok{) }\SpecialCharTok{\%\textgreater{}\%}
\NormalTok{  Seurat}\SpecialCharTok{::}\FunctionTok{CreateAssayObject}\NormalTok{(.)}

\NormalTok{SCpubr}\SpecialCharTok{::}\FunctionTok{do\_ViolinPlot}\NormalTok{(sample, }\AttributeTok{features =} \FunctionTok{c}\NormalTok{(}\StringTok{"OPC{-}like"}\NormalTok{, }\StringTok{"Endothelial"}\NormalTok{), }\AttributeTok{assay =} \StringTok{"paper\_aucell"}\NormalTok{)}
\end{Highlighting}
\end{Shaded}

\includegraphics[width=1\linewidth]{23_Enrich_files/figure-latex/unnamed-chunk-33-1}

\begin{Shaded}
\begin{Highlighting}[]
\NormalTok{SCpubr}\SpecialCharTok{::}\FunctionTok{do\_FeaturePlot}\NormalTok{(sample, }\AttributeTok{assay =} \StringTok{"paper\_aucell"}\NormalTok{, }\AttributeTok{features =}\StringTok{"Astro{-}like"}\NormalTok{)}
\end{Highlighting}
\end{Shaded}

\includegraphics[width=1\linewidth]{23_Enrich_files/figure-latex/unnamed-chunk-33-2}

\end{document}
